\documentclass[12pt,a4paper]{article}
\usepackage[latin1]{inputenc}
\usepackage{amsmath}
\usepackage{amsfonts}
\usepackage{amssymb}
\title{Notes}
\author{Alessandro Pisa}
% Generated by kile and templator
\begin{document}
\section{Brief description of $A_0$}
$A_0$ is the parameter bound to the secondary extinction. It is given by the expression:
\begin{equation}
A_0 = \frac \pi \lambda |{\psi _{hkl}}| \frac {t_0}{\gamma_0}
\end{equation} 
The value of $\psi _{hkl}$ is due mainly to the material characteristics:
\begin{equation}
|\psi _{hkl}| 	= \frac {4\pi  e^2}{m\omega^2} \frac{F_{hkl}}V
= \frac 1\pi r_e \lambda^2 \frac{F_{hkl}}V
\end{equation}

Hence the value of $A_0$ can be expressed as:
\begin{equation}
A_0 = \frac{r_ehc}{V}F_{hkl} \frac{t_0}{E\gamma_0}
\end{equation} 
For crystalline copper with Miller indices $hkl=[1,1,1]$ the values are:
\begin{itemize}
\item[--] $r_e=2.82~10^{-5}$ \AA;
\item[--] $hc=12.4$ keV\AA$^{-1}$;
\item[--] $F_{111}=88.2$;
\item[--] $V=47.0$ \AA$^3$;
\end{itemize}
thus, for this particular choice, this formula can be rewritten as:
\begin{equation}
A_0 = 65.6~10^{-5}\frac{\text{keV}}{\text{\AA}} \frac{t_0}{E\gamma_0}=
6.56\frac{\text{keV}}{\mu\text m} \frac{t_0}{E\gamma_0}
\end{equation} 

We can also define an energy dependent parameter $t_c$:
\begin{equation}
t_c=\frac{EV}{r_e~hc~F_{hkl}}=0.152 \frac{\mu \text m}{keV}~E
\end{equation}

In this way the value of $A_0$ is given by:
\begin{equation}
A_0 = \frac{t_0}{t_c~\gamma_0}
\end{equation}
\end{document}